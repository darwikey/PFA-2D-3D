Explication des tests réalisés pour valider les besoins fonctionnels/non fonctionnels du cahier des charges

\paragraph{}
Pour valider la réalisation des besoins fonctionnels et non fonctionnels présentés dans le cahier des charges, une série de tests a été effectuée sur le logiciel.

\paragraph{}
Concernant les besoins fonctionnels, nous avons manipulé le logiciel sous les différents systèmes d'exploitation envisagés, à savoir Windows et Linux.
Les différentes rotations de la caméra autour de la scène ont bien été implémentées, permettant à l'utilisateur de voir les objets qui y sont placés sous différents angles, et de s'en rapprocher ou de s'en éloigner grâce au zoom.
[TRANSLATION]
Les objets sont issus de fichiers d'extension OBJ version 3.0, ASCII, ou PLY, versions 1.0, ASCII et binare. Ceux-ci peuvent ensuite être sélectionnés un par un, en cliquant sur la scène ou sur son nom dans la liste des objets. L'utilisateur peut alors les déplacer grâce à des translations et rotations selon trois axes, et modifier leur taille. Ils pourront également être supprimés de la scène.
Enfin, les différents rendus prévus ont pu être obtenus à partir de la scène. Pour chacun d'entre eux, une nouvelle fenêtre est ouverte, proposant les paramètres modifiables pour la génération des photographies, anaglyphes, autostéréogrammes, et folioscopes. Pour chacun d'entre eux, on proposera une prévisualisation, et la possibilité d'enregistrer [LES IMAGES INTERMEDIAIRES ayant permis ce résultat, à savoir les vues gauche et droite pour les anaglyphes, la carte des profondeurs pour les autostéréogrammes, et chaque image du folioscope permettant d'obtenir un GIF animé.]    
[VERIFIER SI CEST OK]
[SAUVEGARDES]


\paragraph{}
Les tests des besoins non fonctionnels ont permis de valider la portabilité et la fluidité du logiciel. L'extensibilité du logiciel aura été permise par l'architecture du logiciel, qui a été présentée dans la partie Réalisation du projet.
Pour s'assurer de la portabilité du logiciel, nous avons manipulé le logiciel sous les différentes machines de notre équipe de programmeur, aussi bien sous Linux que sous Windows, avec ou sans carte graphique intégrée. L'ensemble des fonctionnalités précisées dans la paragraphe précédent ont été testées pour s'assurer de leur bon fonctionnement.
La fluidité du logiciel a été testée grâce au modèle happy.ply qui avait servi de modèle lors de la réalisation du cahier des charges. 
[SCREENS]
[RESULTATS WINDOWS/LINUX/CHIPSET]
[LOGICIEL DE TEST]

\paragraph{}
L'ensemble de ces tests auront permis de s'assurer de la conformité de notre projet au Cahier des charges. 
