\paragraph{}
        Le projet PFA a pour objectif de faire découvrir aux élèves la gestion d'un projet depuis la création du cahier des charges jusqu'à l'implémentation en ayant à faire à des clients réels. C'est le premier projet du cursus se réalisant avec un groupe de taille conséquente (en moyenne 7 personnes) et dans lequel les élèves établissent en échangeant avec le client les spécifications. Celui-ci se déroule sur les deux semestres, et se décompose en deux phases. Une première phase durant le premier semestre pour établir les besoins, et les présenter de manière formelle dans le cahier des charges. Puis la deuxième au cours du second semestre pour concrétiser ces besoins en réalisant ce qui est spécifié.

\paragraph{}      
        Le projet PFA effectué : De la 3D à la 2D, est un projet d'imagerie numérique. L'objectif est de réaliser une scène 3D constituait de un ou plusieurs modèles 3D pour ensuite générer des images en deux dimensions de celle-ci permettant de recréer une illusion de trois dimensions en se basant sur des principes connus. Les algorithmes étudiés au cours de ce projet permettent de générer des anaglyphes, autostéréogrammes et folioscopes qui seront présentés plus en détails dans la suite du rapport.
        
\paragraph{}
        Pour commencer le domaine de la synthèse d'image et les différents rendus à implémenter dans le logiciel seront présentés, puis brièvement les besoins visés lors de la phase d'écriture du cahier des charges avant de parler finalement de l'implémentation du projet.
