\paragraph{}
Dans la dernière phase de notre implémentation, nous avons pu ajouter quelques fonctionnalités pour améliorer le logiciel et le rendre plus proche de certains logiciels plus professionnels.

Tout d'abord, à la demande de nos clients, nous avons ajouté la possibilité d'annuler la dernière action effectuée quand celle-ci agit sur l'état d'un objet, son ajout ou sa suppression. Grâce à un stockage de lambda-expression, permises par le C++11, chaque action est enregistrée. Lorsque l'utilisateur utilise le raccourci clavier Control+Z, la dernière fonction ajoutée au vecteur de lambda-expression est récupérée, et l'inverse de l'action est effectuée.

Nous avons également implémenté des fonctionnalités pour coller aux préférences de l'utilisateur. Par exemple, l'utilisateur peut choisir le durée qu'il souhaite imposer entre deux sauvegardes automatiques du logiciel, la couleur du fond de la scène, ou encore l'emplacement de la barre verticale dans laquelle sont listés les objets présents sur la scène. Il peut également modifier les raccourcis clavier pour les personnaliser.

La couleur des objets est également devenue un paramètre modifiable. Après sélection d'un objet, l'utilisateur peut, grâce au menu déroulant prévu à cet effet, en modifier la couleur. Cette modification sera effective dans la scène et lors des rendus, mais ne sera pas sauvegardée avec les autres caractéristiques de l'objet.

Un repère a été placé dans la scène pour que l'utilisateur puisse à tout moment se repérer par rapport aux trois axes géométriques.

Enfin, nous avons mis en place un antialiasing pour la génération des rendus.
[EXPLICATION ANTIALIASING]


\paragraph{}
Faute de temps, d'autres améliorations éventuelles du logiciel n'ont pas pu être mises en place, mais représentent d'éventuelles perspectives d'amélioration pour notre logiciel.

Ce projet a été conçu pour permettre son extensibilité, principalement au niveau des algorithmes et des possibilités de création de rendus divers. On pourrait alors réfléchir à implémenter de nouveaux algorithmes, existants ou à venir, pour tester et comparer leur qualité. On pourrait également imaginer d'autres rendus, comme un flipbook d'anaglyphes ou des autostéréogrammes dynamiques.

Bien que le logiciel permette à l'utilisateur de le personnaliser, certains raccourcis ou fonctionnalités n'ont pas été implémentés, comme la sélection multiple d'objets ou le clic droit sur un objet déjà sélectionné pour connaître d'éventuelles actions effectuables sur cet objet. Cette éventualité pourrait permettre de rendre le logiciel plus agréable à manipuler pour ses utilisateurs.

Un autre objectif potentiel serait le passage du logiciel sous un environnement MAC. Faute de machines de test, nous n'avons pas eu l'occasion de tester notre code avec ce système d'exploitation, mais nous pensons que la transition vers celui-ci pourrait s'avérer relativement simple.

Enfin, l'utilisation de la version 2.0 ES de la bibliothèque OpenGL offre la possibilité d'implémenter le logiciel sous d'autres plateformes, comme par exemple des tablettes graphiques ou des téléphones portables. On pourrait éventuellement imaginer prendre des photos d'une scène environnante et la transformer grâce à d'autres algorithmes en des anaglyphes ou des autostéréogrammes. 

\paragraph{}
Les possibilités d'évolution du logiciel Project3Donuts sont donc multiples, et nous espérons que celui-ci pourra être amélioré ou réutilisé par la suite dans de nouveaux projets.
