\paragraph{}
Au cours de ce projet, certaines difficultés rencontrées auront parfois ralenti l'avancement initialement prévu.

\subsubsection{Implémentation des algorithmes}
\paragraph{}
L'implémentation des algorithmes d'anaglyphes et d'autostéréogrammes était le coeur de notre projet et du logiciel que nous avions à mettre en place, il était donc important que nous lui accordions un temps conséquent durant notre projet. C'est pour cela que nous avions initialement prévu de faire travailler deux personnes sur ces algorithmes durant un mois, en partant des recherches qui avaient été effectuées pour le cahier des charges. Mais cette durée était bien trop courte car elle n'était  pas uniquement dédiée au PFA et nous avons donc du la doubler.

\paragraph{}
Pour les autostéréogrammes, le rendu donné par le premier algorithme choisi était découpé en tranches comme nous pouvoir le voir sur la figure \ref{fig:sphAutoste}.

\begin{figure}[h]               
	\centering      
	\includegraphics[scale=0.7]{bouleAutoste.png}
	\caption{\label{fig:sphAutoste} Autostéréogramme d'une boule avec un effet en tranche \protect}
\end{figure}

Cet effet avait en effet été souligné dans l'article de Witten, Inglis et Thimbleby, mais l'utilisation de points aléatoires pour le fond de l'image accentuait cet effet de tranches. 
Il a alors fallu mettre en place le second algorithme que nous avons présenté précédemment dans ce rapport, ce qui a allongé la période de réalisation. De plus, l'optimisation proposé pour ce deuxième algorithme par W.A.Steer posait dans un premier temps problème pour la génération de l'autostéréogramme. Il aura alors fallu procéder étape par étape pour obtenir dans un premier temps un rendu sans l'optimisation de l'algorithme, qui aura été ajoutée par la suite.

\paragraph{}
Concernant les anaglyphes, en raison de problèmes liés aux shaders sur les processeurs graphiques intégrées au début du projet, nous avons opté pour un prototype utilisant la librairie OpenCV et l'utilisation de pairs d'images sétéréoscopiques trouvés sur Internet pour tester les algorithmes.

Quelques problèmes d'implémentations ont été rencontrés, notamment au niveau de dépassement dans les valeurs RGB des pixels (pixels brulés) dû à des formules mal appliqués ou à une simplification non-voulue d'un flottant en entier. Par exemple, dans l'algorithme de Dubois, il est demandé de vérifier que les valeurs réduites RGB des pixels additionés est bien comprise entre 0 et 1 avant l'addition.

La lecture des articles scientifiques n'a pas été chose aisée. Nien souvent, la description des algorithmes restait floue. Pour l'implémentation de l'algorithme de Dubois nous avons par exemple choisi de n'exploiter qu'une partie de l'article et d'y ajouter une partie sur la correction gamma. D'autres combinaisons ont été faites. Par exemple dans le cas de Dubois, une modification est réalisée sur la saturation sans préciser comment adapter les valeurs RGB en conséquence. Des tests ont été faits avec une conversion de RGB vers HSL sans vraiment produire un résultat correcte.

Pour l'impression des anaglyphes nous nous sommes rapidement rendu compte qu'un résultat de bonne qualité ne pouvait être obtenu que de manière empirique et qu'il n'était pas possible d'implémenter un algorithme permettant d'obtenir de manière général de bons rendus à l'impression quelque soit l'image en entrée, le support en sortie, et le périphérique pour l'impression. C'est pourquoi nous avons préféré laisser à l'utilisateur la possibilité de paramètrer comme bon lui semble le rendu.

\subsubsection{L'utilisation de shaders}
\paragraph{}
On appelle shaders des programmes informatiques qui vont travailler directement sur la carte graphique ou le processeur graphique d'un ordinateur pour permettre un rendu meilleur et plus rapide. Ces shaders sont souvent utilisés en imagerie numérique, notamment par le logiciel Blender\footnotemark qui aura été un des logiciels-modèle pour la création du nôtre. L'utilisation de ces shaders nous paraissait importante pour permettre un rendu agréable dans notre logiciel, et pour atteindre nos objectifs de fluidité énoncés dans notre Cahier des charges.
\footnotetext{\url{http://www.blender.org/}}
\paragraph{}
Dans certains ordinateurs, et surtout dans les ordinateurs portables, il n'existe pas de carte graphique à proprement parler. Un chipset graphique en général moins puissant la remplace. Toutefois, ces chipsets peuvent empêcher l'utilisation de shaders trop récent.

\paragraph{}
Dans une première version de notre logiciel, nous avions souhaité utiliser la version 3.3 des shaders. Toutefois, celle-ci posait problèmes car elles n'étaient pas reconnue par certains chipsets présents sur les ordinateurs portables que nous utilisions. Du fait de problèmes avec la technologie Optimus les machines sous Linux utilise presque exclusivement le chipset intégré, cela posait un problème pour la portabilité. Celle-ci étant primordiale, nous avons dû effectuer une modification de l'ensemble du code déjà construit afin de passer à la version 2.0 des shaders qui est bien plus ancienne et qui permet la portabilité du logiciel sur quasiment toute les machines.

\subsubsection{Les outils utilisés pour la réalisation du projet}
\paragraph{}
Pour permettre la portabilité et la maintenabilité du projet, nous avons choisi de l'implémenter en C++11, en utilisant Qt5 et la bibliothèque OpenGL pour Qt, et en compilant avec l'outil CMake.

\paragraph{}
La communauté Internet sur la bibliothèque Qt et son module OpenGL est très importante et de nombreux problèmes rencontrés ont pu être résolus facilement grâce à des recherches sur certains forums. 
L'outil souvent utilisé dans la réalisation de projet Qt est QMake qui permet une compilation simplifiée avec Qt. Mais nous avons choisi de prendre CMake comme nous savions déjà l'utiliser.
CMake étant très peu utilisé pour la bibliothèque Qt, il est parfois difficile de trouver sur Internet de l'aide pour inclure des modules, par exemple le module QtXml. La plupart de la documentation relative à ce problème est très simplement géré dans QMake, mais il aura fallu beaucoup de recherches pour parvenir à trouver les bons noms de bibliothèques et de modules pour l'inclure avec CMake.

\subsubsection{Portabilité entre systèmes d'exploitation}
\paragraph{}
Tout au long de la réalisation du logiciel, il nous aura souvent fallu corriger des passages de code pour permettre la portabilité aussi bien sous Windows que sous Linux. 
\paragraph{}
Un premier problème de portabilité s'est présenté avec les shaders, comme nous l'avons expliqué plus haut. Les machines portables avec la technologie Optimus rencontre des problèmes de ``switch'' sous linux entre le processeur graphiuque intégré et celui de la carte graphique. Il a fallu utiliser une version plus ancienne pour s'assurer que le logiciel fonctionne. Malgré cette modification, quelques problèmes d'affichage subsistent encore sous Linux, mais ils n'empêchent pas la bonne exécution de l'application.

\paragraph{}
Un autre cas a eu lieu lors de l'utilisation de threads pour la sauvegarde automatique. La fonction sleep() avait d'abord été utilisée, mais celle-ci ne fonctionnait que sous Linux ou avec le compilateur GCC sous Windows. Nous avons donc dû le remplacerpar l'appel de C++ sleep\_for, que l'on trouve dans le namespace this\_thread. 

\paragraph{}
La compatibilité a également été un souci pour le découpage des fichiers de chargement d'extension XML. Sous Windows, le stockage des flottants utilise un point pour séparer la partie entière de la partie décimale, aussi bien en lecture qu'en écriture. Sous Linux, la lecture utilise une virgule. Nous avons donc dû tenir compte de cette différence pour le découpage et l'utilisation de nos fichiers pour le chargement.
