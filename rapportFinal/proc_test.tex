
Settings\\
		Ouvrir le programme / Restaurer les paramètres par défaut\\
		Ouvrir Editer/Préférences\\
		Vérifier que les valeurs ont bien changé.\\
		Dans l'onglet Général, cliquer sur Restaurer les valeurs par défaut\\
		Vérifier que les valeur changent dans l'onglet Général et non dans l'onglet Raccourcis.\\
		Cliquer sur Annuler\\
		Ouvrir Editer/Préférences\\
		Vérifier que les valeurs n'ont pas changé.\\
		Re-faire l'opération pour Raccourcis.\\
\\

Effectuer un rendu\\
		Ouvrir le programme / Restaurer les paramètres par défaut\\
		Ouvrir une scène \\
		Appuyer sur P\\
		Vérifier que la fenêtre de rendu s'affiche\\
		Changer la couleur de fond\\
		Cliquer sur prévisualiser \\
		Vérifier le changement\\
		Changer la correction gamma\\
		Cliquer sur prévisualiser \\
		Vérifier le changement\\
		Faire pareil pour anti-alliasing\\
		Cliquer sur sauvegarder\\
		Dans la nouvelle fenêtre, choisir un dossier et un nom de fichier\\
		Cliquer sur annuler\\
		Vérfier qu'aucun fichier n'a été créé\\
		Recommencer l'opération en acceptant\\
		Vérifier que le fichier a été créé et qu'il correspond bien à ce qui était prévisualisé\\
\\

Anaglyphe, Flipbook, Autostéréogramme\\
		Faire pareil pour tous les algos\\
\\

Déplacement dans la scène / Sélection\\
		Ouvrir le programme / Restaurer les paramètres par défaut\\
		Ouvrir une scène \\
		Vérifier que le mode translation est activé\\
		Sélectionner un objet dans la scène\\
		Vérifier qu'il apparaît en surbrillance\\
		Sélectionner un autre objet dans la scène\\
		Vérifier qu'il apparaît en surbrillance et que l'autre n'est plus sélectionné\\
		Double-cliquer sur un nom d'objet dans la liste\\
		Vérifier qu'il apparait en surbrillance et que l'autre n'est plus sélectionné\\
		Double cliquer sur un autre nom d'objet dans la liste\\
		Vérifier qu'il apparait en surbrillance et que l'autre n'est plus sélectionné\\
		Appuyer sur R\\
		Vérifier que le mode rotation est enclenché et les deux autres désélectionnés\\
		Tester les différentes rotations sur l'objet\\
		Recommencer pour scale et translation\\
		Bouger la camera autour des objets, dans toutes les directions\\
		Déplacer la caméra avec la molette\\
		Cliquer sur la scène avec le bouton central et translater la caméra \\
                Appuyer sur Z pour recentrer la caméra\\
\\
		
Import\\
		Ouvrir le programme / restaurer les parametres par défaut\\
		Ouvrir la fenêtre d'import depuis la bibliotheque, vérifier que le chemin correspond à celui dans settings\\
		Sélectionner un modèle dans un autre dossier, cliquer sur annuler, vérifier que rien ne se passe\\
		Ouvrir la fenêtre d'import depuis la bibliotheque, vérifier que le chemin correspond toujours à celui dans settings\\
		Sélectionner un modèle avec moins de 200000 faces dans un autre dossier, cliquer sur ouvrir, vérifier que le modèle est ajouté à la scène.\\
		Sélectionner un modèle avec plus de 200000 faces dans un autre dossier, cliquer sur ouvrir, vérifier que la fenêtre de demande de réduction de faces s'ouvre et tester dans les deux cas que le modèle est ajouté à la scène.\\
		Dans le cas de la dégradation, vérifier que le modèle ajouté est dégradé. Faire un rendu pour vérifier que le modèle n'est pas dégradé dans le rendu\\
		Ouvrir une scène \\
		Refaire les opérations précédentes en vérifiant que la fenêtre de copie locale s'affiche, et que la copie ne s'effectue que si on appuie sur oui.\\
		Refaire les opérations précédentes avec Importer depuis le disque dur, ignorer la partie vérfication de chemin\\
\\
		
Sauvegarde/chargement\\
		Ouvrir le programme / restaurer les paramètres par défaut\\
		Cliquer sur sauvegarder, vérifier que la fenêtre sauvegarder-sous s'ouvre\\
		Annuler, vérifier qu'aucune sauvegarde ne se fait\\
		Importer un modèle sans le copier\\
		Cliquer sur sauvegarder, vérifier que la fenêtre sauvegarder-sous s'ouvre\\
		Sauvegarder la scène\\
		Vérifier que la scène a maintenant un nom\\
		Fermer le programme\\
		Ouvrir le programme / Restaurer les paramètres par défaut\\
		Ouvrir la scene que vous avez sauvegardé et vérifier qu'elle se charge proprement \\
		Faire ces vérifications pour les modèles copiés localement également\\
		Pour tous les types de modifications possibles (nouveau modèle, translation, rotation, scale), vérifier que la scene entre en mode needsave (titre) et qu'un message de demande de sauvegarde apparaît en tentant de quitter, d'ouvrir ou de créer une nouvelle scène\\
		Vérifier que la sauvegarde enlève ce message\\
		\\

Annuler	\\
	Ouvrir le programme / Restaurer les paramètres par défaut\\
		Importer un modèle sans le copier\\
		Pour tous les types de modifications possibles (translation, rotation, scale), vérifier que Ctrl+z ou l'appui sur l'icône en forme de flèche annule l'action
