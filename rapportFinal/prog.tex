Explications du déroulement de la réalisation
Présentation du Gant prévisionnel
Point de départ
Idée sur la chronologie (pas trop)

\subsection{Choix de programmation pour la réalisation du projet}
\paragraph{}
        Pour la réalisation du projet, nos clients nous proposaient des langages comme C, C++ ou encore Python. C++ nous est apparu comme le choix le plus judicieux pour notre logiciel, afin de faciliter l'utilisation de bibliothèques telles que OpenGL et Qt et de pouvoir bénéficier de l'aide d'une grande communauté internet en cas de difficultés.
        //pourquoi c++ 11 : voir avec xavier
        //pourquoi qt5 : maintenabilité, prendre les bibliothèques les plus récentes pour qu'elles soient le plus durable possible
        //pourquoi opengl es 2 : maintenabilité, extensibilité du projet

\subsection{Création du diagramme de Gantt}
\paragraph{}
        Une fois le cahier des charges mis en place et l'accord des clients donnés, la première étape du projet consistait à réfléchir au diagramme de Gantt prévisionnel pour les trois mois destinés à la réalisation du projet.
        La priorité était donnée aux algorithmes de réalisation des différents rendus espérés grâce au logiciel, et plus particulièrement aux anaglyphes et aux autostéréogrammes. Il fallait revenir sur les algorithmes choisis et présentés dans le cahier des charges, approfondir les recherches pour être sûr de ne passer à côté d'aucun autre algorithme plus performant, et s'approprier le fonctionnement des algorithmes pour les comprendre puis les retranscrire.
        En parallèle, d'autres personnes pouvaient travailler sur une partie plus proche du logiciel finale, à savoir la manipulation de la scène ou encore les parseurs de fichiers pour le chargement des objets.
        Une fois les manipulations premières de la scène effectuées, les travaux prioritaires seraient le chargement et la sauvegarde de la scène grâce à des fichiers d'extension XML, la création de prises de vue simples ou de folioscopes à partir de la scène et leur sauvegarde, et enfin l'assemblage complet du logiciel avec une interface d'utilisation permettant d'utiliser les différents modules.

\paragraph{}
        Un récapitulatif du diagramme de Gantt prévisionnel du projet est donné en annexe.

\paragraph{}
        
