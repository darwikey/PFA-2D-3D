\noindent ===============================================\\
\noindent DEPENDANCES :\\
\noindent ===============================================\\
\noindent Qt 5.3.1 (\url{https://download.qt.io/archive/qt/5.3/5.3.1/})\\

\noindent ===============================================\\
\noindent SOUS WINDOWS\\
\noindent ===============================================\\
\noindent Installer Qt 5.3.1\\

\noindent Lancer CMake-gui\\

\noindent Lors de la configuration, si CMake ne trouve pas les sources de Qt tout seul, il faut lui indiquer le chemin en rentrant dans le champ Qt5OpenGL\_DIR : (* Chemin vers le dossier d'installation de Qt)\\

*/Qt/5.3/*\_opengl/lib/cmake/Qt5OpenGL\\

\noindent Si CMake ne trouve pas le chemin vers Qt5\_Xml il faut rentrer dans le champ Qt5\_Xml\_DIR :\\

* Chemin vers le dossier d'installation de Qt */Qt/5.3/*\_opengl/lib/cmake/Qt5Xml\\

\noindent ===============================================\\
\noindent SOUS LINUX\\
\noindent ===============================================\\
\noindent Installer Qt 5.3.1, CMake \\

\noindent Ajouter à la variable d'environnement PATH la localisation de la librairie. \\

\noindent Créer un dossier build à la racine du projet Project3Donuts/build en s'assurant que CMakeLists.txt est bien à la racine Project3Donuts.\\

\noindent Se placer dans le dossier et faire ``cmake -DCMAKE\_BUILD\_TYPE=Release ..''\\

\noindent Sortir du dossier ``cd ..''\\

\noindent Lancer le logiciel ``./build/Project3Donut''\\

\noindent Si bumblebee et primus sont installés faire pour switcher sur la carte graphique : ``optirun -b primus ./build/Project3Donut''\\

\noindent ===============================================\\
\noindent DOCUMENTATION\\
\noindent ===============================================\\
\noindent La documentation peut être générée en utilisant la commande "make doc"\\
