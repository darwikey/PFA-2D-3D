NOTICE D'UTILISATION DU LOGICIEL PROJECT3DONUTS\\*
===============================================
\\ \\
Le logiciel Project3Donuts est un logiciel de synthèse d'images permettant la visualisation d'une scène en trois dimensions et la génération de photographies, d'anaglyphes, d'autostéréogrammes et de folioscopes à partir d'elle.
\\ \\ \\
Menus et raccourcis clavier
\\ \\
================= Sauvegarde de la scène =================
\\ \\
Nouvelle scène :			Raccourci Cntrl+N     
	       	 			ou     Icône Nouveau     
		 			ou     Menu déroulant Fichier > Nouveau
\\ \\
Ouvrir une scène pré-enregistrée : 	Raccourci Cntrl+O
       	   	 		   	ou      Icône Ouvrir
				   	ou	Menu déroulant Fichier > Ouvrir
		Fichier de chargement acceptés : fichiers OBJ
\\ \\
Enregistrer la scène en cours :       	Raccourci Cntrl+S
	       	     	      		ou 	Icône Enregistrer
					ou 	Menu déroulant Fichier > Enregistrer
		(Pour enregistrer la scène ouverte sous un nouveau nom : Menu déroulant Fichier > Enregistrer Sous)
\\ \\ \\
================= Déplacements autour de la scène =================
\\ \\
Rotation autour de la scène :  	        Clic gauche continu sur la scène et déplacement de la souris
\\ \\
Translation de la caméra :                    Clic continu du bouton du milieu et déplacement de la souris
\\ \\
Recentrage de la caméra sur 
	l'origine de la scène : 			Raccourci Z
\\ \\
Zoom avant/arrière :  	    		Déplacement de molette de la souris
\\ \\ \\
================= Gestion des objets de la scène =================
\\ \\
Ajouter un objet dans la scène : 	Menu déroulant Fichier > Importer
	   	      	       		     (depuis la bibliothèque : bibliothèque de fichiers utilisables proposée par le logiciel
					      depuis le disque dur   : fichier objets de l'utilisateur)
		Fichier objets acceptés : fichier PLY versions 1.0 ASCII et binaire
			       		  fichier OBJ version  3.0 ASCII
\\ \\
Sélection d'un objet :			Clic droit sur un objet
	       	     			ou     Clic gauche sur le nom de l'objet dans la colonne verticale
\\ \\
Modifier la position d'un objet :	APRES SELECTION DE L'OBJET
	    	     	  		Raccourci clavier T
					ou     Icône Translation
					PUIS Clic gauche continu sur les axes et déplacement de la souris
\\ \\
Modifier la rotation d'un objet :	APRES SELECTION DE L'OBJET
	    	     	  		Raccourci clavier R
					ou     Icône Rotation
					PUIS Clic gauche continu sur les axes et déplacement de la souris
\\ \\
Modifier la taille d'un objet :		APRES SELECTION DE L'OBJET
	    	     	  		Raccourci clavier S
					ou     Icône Scale
					PUIS Clic gauche continu sur les axes et déplacement de la souris
\\ \\
Annuler les modifications :		Raccourci Cntrl+Z
	    		  		ou     Icône Annuler
\\ \\
Modifer la couleur d'un objet :		APRES SELECTION DE L'OBJET
	   	   	      		Menu déroulant Editer > Couleur de l'objet
\\ \\ \\
================= Paramètres du logiciel =================
\\ \\
Modifier la couleur du fond de scène :	Menu déroulant Editer > Préférences
\\ \\
Modifier la durée entre deux 
	 sauvegardes automatiques :	Menu déroulant Editer > Préférences
\\ \\
Changement de côté de la barre 
	 verticale :			Menu déroulant Fenêtres > Mettre la fenêtre de visualisation à gauche (option temporaire)
	 	   			ou     Menu déroulant Editer > Préférences (option durable)
\\ \\
Modifier les raccourcis :		Menu déroulant Editer > Préférences
\\ \\
Autres options :			Menu déroulant Editer > Préférences
\\ \\
Quitter le logiciel :			Raccourci Escap
	   	    			ou Icône Quitter classique du système d'exploitation utilisé
					ou Menu déroulant Fichier > Quitter
\\ \\ \\
================= Obtention des rendus =================
\\ \\
Photographie :	  	    	        Raccourci P
	     				ou     Icône Photographie
					ou     Menu déroulant Outils > Effectuer un rendu
\\ \\
Anaglyphe :	  	    	        Raccourci N
	     				ou     Icône Anaglyphe
					ou     Menu déroulant Outils > Anaglyphes
\\ \\
Autostéréogramme :	  	    	Raccourci U
	     				ou     Icône Autostéréogramme
					ou     Menu déroulant Outils > Autostéréogrammes
\\ \\
Flipbook :	  	    	        Raccourci F
	     				ou     Icône Flipbook
					ou     Menu déroulant Outils > Flipbook
