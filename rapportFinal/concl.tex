\paragraph{}
Le déroulement de ce projet nous aura permis étape par étape de prendre part à la réalisation d'un logiciel. Le cahier des charges étant un travail que nous n'avions jamais eu l'occasion de pratiquer, l'implication de nos clients dans notre avancement nous aura permis d'être rigoureux et de comprendre l'intérêt d'un tel contrat entre le client et les programmeurs.
Au cours de la partie programmation, nous avons pu nous rendre compte de la difficulté à estimer la durée d'une tâche. L'importance du cahier des charges nous est là encore apparue, car cette phase de recherche permet de se renseigner sur le domaine et l'existant propre au sujet à traiter, et de choisir d'éventuels algorithmes pour le futur logiciel.

\paragraph{}
Le PFA aura également pour nous été l'occasion de prendre part à la réalisation complète d'un projet informatique, aussi bien sur la forme (interface graphique du logiciel) que sur le fond (scène en trois dimensions et algorithmes des rendus). L'ampleur de l'exercice nous aura permis de faire face à un travail plus proche de ceux que nous aurons à réaliser en entreprise. Il nous aura appris à partir d'une demande, à bâtir le cahier des charges correspondant, puis à construire à partir de rien le logiciel souhaité en respectant ce cahier. Le travail complet de recherche pour les bibliothèques et les méthodes à utiliser aura été enrichissant car il nous aura appris l'auto-formation et la recherche de solutions.

\paragraph{}
Le suivi régulier du projet par nos clients, souhaité par notre équipe dans l'idée de l'utilisation d'une méthode de type agile, nous aura permis au fur et à mesure de présenter l'avancée de notre travail et de corriger les directions prises pour satisfaire leurs souhaits, ou pour ajouter d'éventuels fonctionnalités comme l'annulation de la dernière action effectuée.

\paragraph{}
Le PFA a donc été une expérience très enrichissante puisqu'elle aura permis de travailler dans des conditions proches de celles que nous pourrions avoir en entreprise.
