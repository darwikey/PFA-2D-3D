\paragraph{}
        Dans le cadre du projet PFA, la première partie consistait à rédiger un cahier des charges fonctionnel pour définir l'ensemble des besoins du futur logiciel.

\paragraph{}
        Lors de notre premier rendez-vous, notre client a d'ores et déjà défini le format qu'il souhaitait pour le cahier des charges, et le contenu qui était essentiel pour lui. Il fallait ainsi présenter le domaine d'études et les connaissances actuelles sur des algorithmes de création d'anaglyphes, d'autostéréogrammes ou de folioscopes. Ensuite, le sujet était à redéfinir précisément, puis les besoins fonctionnels et non fonctionnels demandés par les clients pour ce logiciel, ainsi que les contraintes engendrées par ceux-ci. Enfin, il fallait présenter des prototypes permettant de répondre aux différents besoins ennoncés, quelques interfaces graphiques pour simuler l'utilisation du logiciel, et surtout réfléchir à l'architecture future du projet.

\paragraph{}
        La rédaction des parties Domaine et Etat de l'existant aura demandé de nombreuses recherches, notamment pour trouver des articles présentant des algorithmes de création d'anaglyphes et d'autostéréogrammes. 
L'étude du domaine aura permis de se familiariser avec le vocabulaire de la synthèse d'image et avec les différents rendus souhaités par nos clients. Des notions telles que la scène, la caméra, ainsi que la compréhension du fonctionnement des anaglyphes et des autostéréogrammes, auront ainsi permis une meilleure immersion dans notre projet.
Les algorithmes relatifs aux anaglyphes concernent principalement le traitement des couleurs pour qu'aucun artefact n'apparaisse au moment de la visualisation finale. Pour les autostéréogrammes, le traitement d'une carte des profondeurs permet d'obtenir une image qui, lorsque l'on sait l'observer, fait apparaître un relief. Enfin, il n'existe pas d'algorithme particulier pour la génération de folioscopes. Seule une série de prises de vue d'une scène avec des angles d'observation proches peuvent permettre, si elles sont visualisées les unes à la suite des autres et suffisamment rapidement, de pouvoir imaginer un mouvement, et ainsi un relief.

\paragraph{}
        Les besoins fonctionnels et non fonctionnels d'un projet doivent impérativement être ciblés durant la phase de cahier des charges, car c'est grâce à eux que le contrat passé entre les clients et l'équipe de programmeurs pourra être exhaustif et protéger les deux parties contre d'éventuelles envies de modification au cours de la phase de réalisation. 
Pour un tel logiciel, les besoins fonctionnels s'apparentent bien souvent à de futures fonctionnalités du logiciel, tel que la création d'une scène, sa manipulation, et les prises de vue pour obtenir les rendus demandés dans le sujet. 
Toutefois, les besoins non fonctionnels, même s'ils peuvent être clairement énoncés par le client, sont bien souvent implicites et doivent être déterminés en fonction du discours tenu. Pour un logiciel de synthèse d'images, la fluidité d'affichage est bien souvent un besoin essentiel, car l'utilisateur s'attend à une certaine rapidité du logiciel lors de la manipulation de la scène et de la génération de rendus. Mais d'autres besoins, à savoir la portabilité du logiciel et sa maintenabilité, ont également été exprimés par les clients et ont été considérés comme essentiels pour ce projet. En effet, l'utilisaton de ce logiciel, au moins sous les systèmes d'exploitation Linux et Windows, était importante pour pouvoir travailler sur différentes machines dans leur travail. De plus, un tel logiciel pouvait être amené à être amélioré ou réutilisé en partie dans de futurs projets, et c'est pourquoi il devait être facilement maintenable.
Ces besoins ont apportés des contraintes quand à la réalisation du logiciel, par exemples sur les langages et les bibliothèques utilisées. Les choix ainsi effectués pour permettre de tenir ces contraintes seront détaillées par la suite.

\paragraph{}
        //passer vite fait sur les proto et l'archi
