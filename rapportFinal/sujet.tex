\paragraph{}
	Le projet concerne la réalisation d’un logiciel permettant d’obtenir des projections en deux dimensions, des anaglyphes, des autostéréogrammes ou encore des flipbooks à partir de scènes virtuelles en trois dimensions. 

\paragraph{}
	L’objectif premier est de permettre la visualisation, sur un support en deux dimensions tel qu’un écran d’ordinateur ou une feuille de papier, d’un espace en trois dimensions. A partir de la visualisation d’objets 3D dans une scène il faudra donc réaliser des photographies qui une fois traitées donneront lieu à des anaglyphes, autostéréogrammes ou des animations type flipbook.
	
\paragraph{}
	Afin d’atteindre cet objectif, un logiciel s’appuyant sur le moteur 3D OpenGL devra être réalisé. Il permettra la création d’une scène où s’inséreront des objets dont la position, la taille et l’orientation seront paramétrables. Une caméra permettra de se déplacer dans la scène, de s’en rapprocher ou s’en éloigner.

\paragraph{}
	Une fois la scène mise en place, il faudra pouvoir prendre des photographies de celle-ci sous différents angles afin d’obtenir, après application d’algorithmes de traitement d’images :

\begin{itemize}
	\item
		des anaglyphes rouge-cyan, qui permettront une visualisation en trois dimensions grâce à des lunettes adaptées ;
	\item
		des stéréogrammes, qui sont des images dissimulant un contenu qui apparaît quand on fixe le dessin de façon spécifique ;
	\item
		des flipbooks ou images animées, correspondant à une succession d’images suivant une trajectoire qui permettent en les faisant défiler de donner une impression de mouvement.
\end{itemize}
