\paragraph{}
Ce projet a été pour nous la première occasion de rédiger un cahier des charges formel. Dès notre premier rendez-vous, nos clients ont clairement exprimés leurs attentes vis-à-vis du contenu du cahier des charges. Ils nous ont également permis de comprendre que l'exhaustivité des besoins exprimés permet d'assurer le contrat entre l'équipe des programmeurs et les clients et de certifier à la fois un engagement pour un produit minimal et pour que les demandes vis-à-vis de celui-ci ne puissent pas être revisitées.

\paragraph{}
Nos clients étant d'ores et déjà fixés sur le contenu qu'ils souhaitaient pour le logiciel, il a été assez facile pour nous de rédiger les besoins fonctionnels et non fonctionnels qui avaient été clairement énoncés. La principale difficulté que nous avons rencontré lors de la phase du cahier des charges a été la rédaction des parties Domaine et Etat de l'existant.
La partie Domaine nous aura permis de nous familiariser avec le vocabulaire de la scène et des objets, et de revenir sur l'historique de la synthèse d'image. Toutefois, il nous a fallu faire le tri des informations importantes pour la bonne compréhension du sujet par la suite.
Les recherches relatives à l'Existant étaient d'autant plus importantes qu'elles allaient permettre de choisir des algorithmes les plus performants pour l'obtention des rendus souhaités.

\paragraph{}
Les algorithmes relatifs aux anaglyphes concernent principalement le traitement des couleurs pour qu'aucun artefact n'apparaisse au moment de la visualisation finale. Pour les autostéréogrammes, le traitement d'une carte des profondeurs permet d'obtenir une image qui, lorsque l'on sait l'observer, fait apparaître un relief. Enfin, il n'existe pas d'algorithme particulier pour la génération de folioscopes. Seule une série de prises de vue d'une scène avec des angles d'observation proches peuvent permettre, si elles sont visualisées les unes à la suite des autres et suffisamment rapidement, de pouvoir imaginer un mouvement, et ainsi un relief.

\paragraph{}
La rédaction du cahier des charges aura été un exercice enrichissant car il nous aura appris son importance, et il nous aura entraîné à la recherche d'algorithmes existants pour éviter de devoir repartir de zéro à chaque nouveau projet. Ce document est réellement un contrat destiné à protéger à la fois les clients et les programmeurs, et l'exhaustivité des données qu'il contient est importante pour qu'aucune des deux parties ne puisse changer d'avis au moment de la partie Réalisation, du moins pas sans un accord commun.
