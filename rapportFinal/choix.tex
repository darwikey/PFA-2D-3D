\paragraph{}
        Pour la réalisation du projet, nos clients nous proposaient des langages comme C, C++ ou encore Python. C++ nous est apparu comme le choix le plus judicieux pour notre logiciel, afin de faciliter l'utilisation de bibliothèques telles que OpenGL et Qt et de pouvoir bénéficier de l'aide d'une grande communauté internet en cas de difficultés.
        //pourquoi c++ 11 : voir avec xavier

        Comme nous l'avons exprimé plus haut, l'un des besoins non fonctionnels primordiaux de notre projet est la maintenabilité du code. Pour celà, il nous fallait choisir des outils et des bibliothèques qui étaient destinées à perdurer le plus longtemps possible.
        L'utilisation de Qt et OpenGL pour Qt est assez répandue pour des logiciels tel que le nôtre de visualisation et de manipulation en trois dimensions, et également très complètes car Qt propose un vaste choix de modules qui permettent de traiter un ensemble très variable de tâches (notamment le parsage de fichier XML que nous avons utilisé). Comme ces bibliothèques sont de plus portables, elles convenaient parfaitement à l'implémentation que nous souhaitions réaliser.
        La version 5 de Qt est la plus récente actuellement, mais malgré sa jeunesse elle est devenu au fil des modifications suffisament stable pour pouvoir être utilisée sans problème, et elle dispose d'une communauté Internet active prête à aider en cas de difficultés de code. De plus, cette bibliothèque utilise la version ES 2.0 de OpenGL, qui est une version non seulement qui utilise des shaders, mais également qui peut être utilisée pour de la programmation d'applications mobile par exemple. C'est donc pour cette variété, cette communauté, cette maintenabilité et cette extensibilité que nous avons choisi d'utiliser ces bibliothèques et ces versions.
