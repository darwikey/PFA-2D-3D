\subsection{Superposition des objets dans la scène}

\paragraph{}
	Pour faciliter le traitement d’une scène, on considèrera que deux objets placés dans une scène ne pourront pas être superposés. On parle ici d’intersection des boîtes englobant des objets et non de la gestion des parties non visibles des objets lors de l’affichage.

\subsection{Utilisation des bibliothèques}

\paragraph{}
	Afin de minimiser les dépendances et valider le besoin en portabilité les bibliothèques Qt et OpenGL seront utilisées. CMake permettra de générer les scripts de compilation quel que soit la plateforme.
	
\subsection{Langage de programmation}

\paragraph{}
	Le langage C++ sera utilisé.

\subsection{Open Source}

\paragraph{}
	Pour permettre l’utilisation du logiciel et son extensibilité à quiconque souhaiterait le modifier, on choisira de laisser le code du projet en Open Source. Pour garantir cela, on s’assura bien que l’on dispose des droits nécessaires pour n’importe quel algorithme ou fragment de code utilisé au cours de ce projet.
	