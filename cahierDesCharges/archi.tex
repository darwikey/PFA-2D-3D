Dans cette partie, nous présenterons un ensemble de diagrammes de séquences et de diagrammes de classes présentant le fonctionnement prévisionnel du logiciel et l’architecture qui en découle. Cette architecture tiendra compte du besoin d’extensibilité du logiciel.

\paragraph{}
Puisque le logiciel disposera d’une interface graphique, c’est par elle que passeront toutes les demandes de l’utilisateur et les rendus qu’il demande.

\paragraph{}
L’ensemble des diagrammes de séquence et des diagrammes de classe présentés par la suite auront été créés grâce à la version d’essai du logiciel en ligne Gliffy\footnote{\url{http://www.gliffy.com/}}.


\subsection{Diagramme de séquences}

Les diagrammes de séquence permettent de simuler les différentes communications qui auront lieu entre les classes du logiciel.

