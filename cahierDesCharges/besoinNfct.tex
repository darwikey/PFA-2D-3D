\subsection{La fluidité d'affichage}
\paragraph{}
	Pour que l’affichage soit fluide la machine devra avoir au minimum 2Go de mémoire vive, un processeur récent avec au moins 2 cœurs cadencés au minimum à 2GHz et les conditions suivantes devront être validées :

	\begin{itemize}
		\item 
		Pour une machine avec processeur graphique type Intel HD Graphics 4000 ou plus récent, au maximum 100 000 points pour un affichage à 25 fps, ou 150 000 points pour un affichage à 20 fps ;
		\item
		Pour une machine avec une carte graphique type NVIDIA GTX 720 ou plus récente, au maximum 350 000 points pour un affichage à 25 fps, ou 500 000 points pour un affichage à 20 fps.
		
	\end{itemize}
	
\textbf{\underline{Test de validation}} : ouverture de plusieurs modèles 3D, présentés en annexe, dans Meshlab sur différentes machines. Les résultats sont également disponibles en annexe.

\subsection{Fluidité d'obtention des rendus}
\paragraph{}
	A l’heure actuelle on ne peut garantir un nombre de points assurant l’obtention d’un rendu en un temps raisonnable. C’est pourquoi une barre de progression s’affichera pour que l’utilisateur ait une idée du temps d’attente avant l’affichage du résultat.

\subsection{Portabilité du logiciel}
\paragraph{}
La portabilité rend le logiciel accessible à un plus grand nombre de personnes. Les systèmes d’exploitation visés seront Windows XP, Windows Seven, Windows 8.1, Ubuntu 14.04, Fedora 20 et Debian Jessie, en plateforme 32 bits et 64 bits. \newline

\textbf{\underline{Test de validation}} : un prototype, qui pourra être utilisé comme base du projet, a été exécuté sur l’ensemble de ces systèmes d’exploitation (cf.  la partie \ref{sec:prototypetest} Prototype test).

\subsection{Extensibilité du logiciel}
\paragraph{}
	Le client souhaite que l’extensibilité du logiciel soit facilitée par son architecture. En effet, il envisage par la suite d’ajouter de nouvelles fonctionnalités ou de nouveaux algorithmes au logiciel, et demande à ce que de telles modifications soient aisément envisageables.
\paragraph{}
	La facilitation de cette extensibilité sera permise par l’utilisation d’une architecture spécifique. Celle-ci est détaillée dans la partie \ref{sec:archi} Architecture du projet de ce cahier des charges.   
