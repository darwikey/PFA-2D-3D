\subsection{Meshlab}

\paragraph{}
	En développement depuis 2005, Meshlab est un logiciel libre de traitement de maillage 3D disponible sur Microsoft Windows, Mac OS X, GNU / Linux et ses dérivés. Il propose différents outils comme des filtres de nettoyage du maillage, la possibilité de supprimer le bruit etc. C’est un logiciel très apprécié dans le domaine universitaire pour la reconstruction de surface, l’impression 3D etc. car il s’agit d’un logiciel non seulement gratuit, mais également Open Source. Cela permet d’avoir une base commune et grâce à une communauté active, de nouvelles fonctionnalités peuvent être ajoutées et utilisées par tout le monde. 

\subsection{Blender}

\paragraph{}
	Blender est, depuis 2003, un logiciel libre de modélisation, d’animation et de rendu 3D disponible sur de nombreuses plateformes comme Microsoft Windows, Mac OS X, GNU / Linux et ses dérivés. C’est un logiciel extensible,  léger, ne réclamant pas un ordinateur très puissant pour des actions simples. Sa communauté de fan très développée a permis la réalisation d’une base de données de greffons conséquente. Preuve de sa puissance, celui-ci a été utilisé pour réaliser entièrement un film d’animations (Tears of Steel) et pour les effets spéciaux de plusieurs films tels que Mr. Nobody.
	