%%%%%%%%%%%%%%%%%%%%%%%%%%%%%%%%%%%%%%%%%
% Beamer Presentation
% LaTeX Template
% Version 1.0 (10/11/12)
%
% This template has been downloaded from:
% http://www.LaTeXTemplates.com
%
% License:
% CC BY-NC-SA 3.0 (http://creativecommons.org/licenses/by-nc-sa/3.0/)
%
%%%%%%%%%%%%%%%%%%%%%%%%%%%%%%%%%%%%%%%%%

%----------------------------------------------------------------------------------------
%	PACKAGES AND THEMES
%----------------------------------------------------------------------------------------

\documentclass{beamer}

\usepackage[utf8]{inputenc}
\usepackage[french]{babel}
\usepackage{enumerate}
\usepackage{array,multirow,colortbl}
\usepackage{enumitem}
\usepackage{animate}
\usepackage{graphicx} % Allows including images
\usepackage{booktabs} % Allows the use of \toprule, \midrule and \bottomrule in tables
\usepackage{cite}
\usepackage[numbers]{natbib}
\usepackage[T1]{fontenc}
\usepackage[default]{gillius}
\mode<presentation> {

% The Beamer class comes with a number of default slide themes
% which change the colors and layouts of slides. Below this is a list
% of all the themes, uncomment each in turn to see what they look like.

%\usetheme{default}
%\usetheme{AnnArbor}
%\usetheme{Antibes}
%\usetheme{Bergen}
%\usetheme{Berkeley}
%\usetheme{Berlin}
%\usetheme{Boadilla}
%\usetheme{CambridgeUS}
%\usetheme{Copenhagen}
%\usetheme{Darmstadt}
%\usetheme{Dresden}
%\usetheme{Frankfurt}
%\usetheme{Goettingen}
%\usetheme{Hannover}
%\usetheme{Ilmenau}
%\usetheme{JuanLesPins}
%\usetheme{Luebeck}
%\usetheme{Madrid}
%\usetheme{Malmoe}
%\usetheme{Marburg}
%\usetheme{Montpellier}
%\usetheme{PaloAlto}
%\usetheme{Pittsburgh}
%\usetheme{Rochester}
\usetheme{Singapore}
%\usetheme{Szeged}
%\usetheme{Warsaw}

% As well as themes, the Beamer class has a number of color themes
% for any slide theme. Uncomment each of these in turn to see how it
% changes the colors of your current slide theme.

%\usecolortheme{albatross}
%\usecolortheme{beaver}
%\usecolortheme{beetle}
%\usecolortheme{crane}
%\usecolortheme{dolphin}
%\usecolortheme{dove}
%\usecolortheme{fly}
\usecolortheme{lily}
%\usecolortheme{orchid}
%\usecolortheme{rose}
%\usecolortheme{seagull}
%\usecolortheme{seahorse}
%\usecolortheme{whale}
%\usecolortheme{wolverine}

%\setbeamertemplate{footline} % To remove the footer line in all slides uncomment this line
\setbeamertemplate{footline}[page number] % To replace the footer line in all slides with a simple slide count uncomment this line

\setbeamertemplate{navigation symbols}{} % To remove the navigation symbols from the bottom of all slides uncomment this line
}


%----------------------------------------------------------------------------------------
%	TITLE PAGE
%----------------------------------------------------------------------------------------

\title[PFA 2014-2015]{PFA 2014-2015 \\ \LARGE De la 3D vers la 2D} % The short title appears at the bottom of every slide, the full title is only on the title page

\author{Anaïs BOHER, Yohann CABON, Magali CHAUVAT, Akané LEVY, Thomas MARCELIN, Xavier MAUPEU, Alexandre PHILIPPI
\\ \and \\
\underline{Clients :} Carole BLANC, Pascal DESBARATS 
\\
\underline{Encadrant :} Sylvain LOMBARDY} 

\institute[Enseirb-Matmeca] % Your institution as it will appear on the bottom of every slide, may be shorthand to save space
{École Nationale Supérieure d'Électronique, Informatique, Télécommunications, Mathématique et Mécanique de Bordeaux
}
\date{} %2014-2015} % Date, can be changed to a custom date

\setbeamertemplate{section in head/foot}{\hfill\insertsectionheadnumber.~\insertsectionhead}
\setbeamertemplate{section in head/foot shaded}{\color{structure!50}\hfill\insertsectionheadnumber.~\insertsectionhead}
\setbeamertemplate{section in toc}{\inserttocsectionnumber.~\inserttocsection}

\begin{document}
\graphicspath{{./images/}{.}}


%------------------------------------------------



\begin{frame}
\titlepage % Print the title page as the first slide

\begin{figure}[B]
\vspace*{-1cm}
\includegraphics[scale=0.4]{logo.png}
\end{figure}
\end{frame}
% % TODO : caption pour les images 

%------------------------------------------------

%plan sur le cote 
%introduction
%cahier des charges 
%implementation
%gestion des projets

%------------------------------------------------
\section{Introduction}

% TODO : remplir la slide du contexte
\begin{frame}
\frametitle{Contexte}

\begin{itemize}[label=$\bullet$]
\item Projet proposé par des chercheurs du LaBRI
\item parler du domaine ?
\end{itemize}
\end{frame}

%------------------------------------------------

% %TODO : remplacer par un schéma, les points 
% TODO objectif: représentation d'une scène
%parler des anaglyphes, autostéreogrammes, flipbooks 
%slide 1:  schéma 
% slide d'après : anaglyphes/autostereogrammes/gif/scene

\begin{frame}
\frametitle{Présentation du sujet}

\begin{itemize}[label=$\bullet$]
\item Schéma A FAIRE
\end{itemize}
\end{frame}

%------------------------------------------------

% % TODO : MAGALI tu peux vérifier le caption et mettre ce qu'il faut, et remplacer l'image aussi
\begin{frame}

\frametitle{Exemple d'autostéréogramme}
\begin{figure}
\centering
\includegraphics[scale=0.4]{autostereogramme1texture.png} % de même
\caption{Autostéréogramme d'une scène contenant deux tores}
\end{figure}

\end{frame}

%------------------------------------------------

\begin{frame}

\frametitle{Exemple d'anaglyphe}
\begin{figure}
\centering
\includegraphics[scale=0.38]{donuts_photoshop.png} % pas obligé de mettre des donuts
\caption{Anaglyphe d'une scène contenant deux tores}
\end{figure}

\end{frame}

%------------------------------------------------

% % TODO changer par un gif ici, pas obligé de mettre les donuts
% % TODO : faire apparaître le mot "folioscope"


\begin{frame}

\frametitle{Exemple de flipbook/ folioscope}

\begin{figure}
\centering
\animategraphics[autoplay,loop,height=6cm]{1}{gifimages/rendu_gif_}{00}{49}
\caption{Animation gif avec un mouvement de rotation autour d'une scène contenant deux tores}
\end{figure}
\end{frame}

%------------------------------------------------
% TODO : les besoins en liste 
%
\section{Cahier des charges}


\begin{frame}
\frametitle{Besoins non fonctionnels et contraintes}
\begin{itemize}[label=$\bullet$]
\item Besoins fonctionnels
	\begin{itemize}[label=$\circ$]
	\item Manipulation de la scène
	\item Manipulation des objets
	\item Sauvegardes automatiques et manuelles, chargement
	\item Anaglyphes, autostéréogrammes, flipbook
	\end{itemize}
\item Besoins non fonctionnels
	\begin{itemize}[label=$\circ$]
	\item Extensibilité
	\item Portabilité
	\item Fluidité
	\end{itemize}
\end{itemize}
\end{frame}

%------------------------------------------------
% ensuite présenter archi  
% ATTENTION pas de capture d'écran ici
% TODO : présenter l'architecture qui remplie les besoins du cahier de charge
% classes virtuelles pour l'extensibilité par ex

\begin{frame}
\frametitle{Architecture}
\begin{itemize}[label=$\bullet$]
\item Extensibilité : classes virtuelles  
\end{itemize}
\centering
\includegraphics[scale=0.25]{extensibilite.png}
\end{frame}

%------------------------------------------------

%\begin{frame}
%\frametitle{Besoins non fonctionnels et contraintes}
%\begin{itemize}[label=$\bullet$]
%\item Portabilité
%\end{itemize}
%\centering
%\includegraphics[scale=0.7]{portabilite.png}
%\\
%Même screen sous Linux 
%
%\end{frame}

%------------------------------------------------

%\begin{frame}
%\frametitle{Besoins non fonctionnels et contraintes}
%\begin{itemize}[label=$\bullet$]
%\item Fluidité
%\end{itemize}
%\centering
%\includegraphics[scale=0.7]{fluidite.png}
%\\
%Même screen avec cg 
%
%\end{frame}
%%------------------------------------------------

%%------------------------------------------------
% TODO : au début présentation du logiciel, faire une scène 2 tores, en rendu, flipbook, anaglyphes, autostereogrammes, 
% que le résultat

\section{Implémentation}

\begin{frame}
\frametitle{Présentation du logiciel}
\begin{figure}
\includegraphics[scale=0.22]{logiciel.png}
\caption{Vue du logiciel}
\end{figure}
\end{frame}
% TODO : couper les screens et les remplir
% TODO : mettre des sous titres

%%------------------------------------------------
%
% TODO : MAGALI screen shot plz
\begin{frame}
\frametitle{Présentation du logiciel : rendu de la scène}
\begin{figure}
\centering
\includegraphics[scale=0.3]{rendu.png}
\caption{Rendu d'une scène contenant deux tores}
\end{figure}

\end{frame}

%%------------------------------------------------
%
\begin{frame}
\frametitle{Présentation du logiciel : folioscope}
\begin{figure}
\centering
\animategraphics[autoplay,loop,height=6.5cm]{1}{gifimages/rendu_gif_}{00}{49}
\caption{Gif animé obtenu à partir du logiciel}
\end{figure}
\end{frame}



%%------------------------------------------------
%
% TODO : MAGALI screen shot plz
\begin{frame}
\frametitle{Présentation du logiciel : autostéréogramme}
\begin{figure}
\centering
\includegraphics[scale=0.31]{renduautostereogrammes.png}
\caption{Autostéreogramme obtenu à partir du logiciel}
\end{figure}
\end{frame}

%%------------------------------------------------
%
% TODO : MAGALI screen shot plz
\begin{frame}
\frametitle{Présentation du logiciel : anaglyphe}
\begin{figure}
\centering
\includegraphics[scale=0.31]{renduanaglyphe.png}
\caption{Anaglyphe obtenu à partir du logiciel}
\end{figure}
\end{frame}

%%------------------------------------------------

% % XAVIER :

\begin{frame}
\frametitle{Choix techniques}
\begin{itemize}[label=$\bullet$]
	\item Utilisation du C++11
	\item Interface graphique avec Qt5
	\item Rendu avec OpenGL
\end{itemize}

\end{frame}

%------------------------------------------------

\begin{frame}
\frametitle{Rendu 2D avec OpenGL}
\begin{figure}
\centering
\includegraphics[scale=0.15]{rendu_specular.png}
\caption{Rendu avec et sans spécularité}
\end{figure}

\end{frame}

%------------------------------------------------

\begin{frame}
\frametitle{Anti-aliasing}
\begin{figure}
\centering
\includegraphics[scale=0.8]{antialiasing.png}
\caption{Rendu avec et sans antialiasing}
\end{figure}

\end{frame}

%------------------------------------------------

%
\begin{frame}
\frametitle{Autostéréogramme : principe}

\begin{itemize}[label=$\bullet$]
\item Génération d'une carte de profondeurs
\item Obtention de l'autostéréogramme
\end{itemize}

\begin{figure}
\centering
\includegraphics[scale=0.22]{donutdepth.png}
\caption{Carte des profondeurs obtenue avec le logiciel}
\end{figure}
\end{frame}
%%------------------------------------------------
% ORAL : demande une certaine gymnastique de voir les autostéréogrammes
% recherche existant , choix d'implémenter deux pour avoir deux rendus différents
\arrayrulecolor{white}   

\begin{frame}
\frametitle{Autostéréogramme}
\begin{itemize}[label=$\bullet$]
\item Algorithme de Thimbelby, Inglis, Witten \cite{stereogram}
	\begin{itemize}[label=$\circ$]
	\item Algorithme ``de base''
	\item Plus adapté aux formes simples et peu détaillés
	\end{itemize}
\end{itemize}

\begin{tabular}{l|r}
\centering
\includegraphics[scale=0.22]{donutdepth.png}
&
\centering
\includegraphics[scale=0.22]{donut1.png}
\end{tabular}

\end{frame}

%%------------------------------------------------
%
% TODO : MAGALI complexite à rajouter
\begin{frame}
\frametitle{Autostéréogramme}
\begin{itemize}[label=$\bullet$]
	\item Algorithme de W.A Steer \cite{wasteer}
	\begin{itemize}[label=$\circ$]
	\item Algorithme plus élaboré
	\item Amélioration 3D : suréchantillonage
	\item Amélioration 2D : coloration symétrique
	\item Complexité : $O(mn)$
	\end{itemize}
\end{itemize}
\begin{tabular}{l|r}
\includegraphics[scale=0.22]{donutdepth.png}
&
\includegraphics[scale=0.22]{donut2.png}
\end{tabular}

\end{frame}
%%------------------------------------------------

%

% % ORAL : on avait deux images de base on les a pas généré ! 
\begin{frame}
\frametitle{Anaglyphe}
\begin{itemize}[label=$\bullet$]
\item Méthode Photoshop \cite{stereoAnaglyph}
	\begin{itemize}[label=$\circ$]
	\item Algorithme ``de base''
	\item Composé de trois étapes 
	\item Première étape : translation de la prise de vue d'une distance proche à celles entre les yeux 
	\end{itemize}
\end{itemize}
\begin{tabular}{l|r}
\includegraphics[scale=0.15]{flip1.png}
vue gauche
&
\includegraphics[scale=0.15]{flip2.png}
vue droite
\end{tabular}

\end{frame}
%%------------------------------------------------
%
\begin{frame}
\frametitle{Anaglyphe}
\begin{itemize}[label=$\bullet$]
\item Méthode Photoshop \cite{stereoAnaglyph}
	\begin{itemize}[label=$\circ$]
	\item Deuxième étape : extraction du canal rouge pour la vue gauche, canaux verts et bleus pour la vue droite 
	\end{itemize}
\end{itemize}
\begin{tabular}{l|r}
\includegraphics[scale=0.15]{flip1.png}
&
\includegraphics[scale=0.15]{flip2.png}
\end{tabular}

\end{frame}

%%------------------------------------------------
%
\begin{frame}
\frametitle{Anaglyphe}
\begin{itemize}[label=$\bullet$]
\item Méthode Photoshop \cite{stereoAnaglyph}
	\begin{itemize}[label=$\circ$]
	\item Troisième étape : reconstitution des canaux RGB à partir des deux images
	\end{itemize}

\begin{figure}
\centering
\includegraphics[scale=0.25]{donuts_photoshop.png}
\caption{Anaglyphe avec la méthode Photoshop d'un objet simple ayant peu de détails }
\end{figure}
	\begin{itemize}[label=$\circ$]
	\item Méthode efficace pour des objets simples ayant peu de détails
	\end{itemize}
\end{itemize}

\end{frame}


%%------------------------------------------------
%
\begin{frame}
\frametitle{Anaglyphe}
\begin{itemize}[label=$\bullet$]
\item Méthode Photoshop \cite{stereoAnaglyph}
	\begin{itemize}[label=$\circ$]
	\item Méthode moins efficace pour les objets complexes 
	\item Plusieurs artefacts visibles
	\item Les détails ne sont pas soignés
	\end{itemize}
\begin{figure}
\centering
\includegraphics[scale=0.3]{happy_photoshop.png}
\caption{Anaglyphe avec la méthode Photoshop d'un objet complexe ayant beaucoup de détails }
\end{figure}
	
\end{itemize}

\end{frame}
% TODO : rajouter slide : mettre pour happy avec photoshop : marche moins bien 

%%------------------------------------------------
%
% ORAL : on a continué notre recherche
% les vraies deux images décalées
\begin{frame}
\frametitle{Anaglyphe}
\begin{itemize}[label=$\bullet$]
\item Méthode Dubois \cite{algoDubois}
	\begin{itemize}[label=$\circ$]
	\item Combinaison de couleurs plus élaborée
	\item Conservation des détails de l'objet dans l'anaglyphe 
	\end{itemize}
\end{itemize}
\begin{figure}
\centering
\includegraphics[scale=0.3]{happy_dubois.png}
\caption{Anaglyphe avec l'algorithme de Dubois d'un objet complexe ayant beaucoup de détails}
\end{figure}

\end{frame}

%%------------------------------------------------
%
% 
\begin{frame}
\frametitle{Anaglyphe}
\begin{itemize}[label=$\bullet$]
\item Méthode Dubois \cite{algoDubois}
	\begin{itemize}[label=$\circ$]
	\item Légèrement moins efficace pour les objets simples : 
	\end{itemize}
\end{itemize}
\begin{figure}
\centering
\includegraphics[scale=0.35]{happy_dubois.png}
\caption{Anaglyphe avec l'algorithme de Dubois d'un objet simple avec peu de détails}
\end{figure}
\end{frame}
%------------------------------------------------

\begin{frame}
\frametitle{Besoins non fonctionnels}
\begin{itemize}[label=$\bullet$]
 	\item Portabilité : Linux (debian Jessie, Fedora, ArchLinux), Windows (Windows 7, Windows 8, )
	\item Fluidité : 
\end{itemize}

\end{frame}
% TODO fluidite : mettre un tableau

%------------------------------------------------

\begin{frame}
\frametitle{Bilan technique}
\begin{itemize}[label=$\bullet$]
\item Besoins fonctionnels
	\begin{itemize}[label=$\checkmark$]
	\item Manipulation de la scène
	\item Manipulation des objets
	\item Sauvegardes automatiques et manuelles, chargement
	\item Anaglyphes, autostéréogrammes, flipbook
	\end{itemize}
\item Besoins non fonctionnels
	\begin{itemize}[label=$\checkmark$]
	\item Extensibilité
	\item Portabilité  
	\item Fluidité % TODO : mettre tableau avec nombre de FPS une slide à rajouter 
	\end{itemize}
\end{itemize}

\end{frame}

%%------------------------------------------------%

\begin{frame}
\frametitle{Difficultés rencontrées}
\begin{itemize}[label=$\bullet$]
\item Difficultés techniques
\begin{itemize}[label=$\circ$]
\item Utilisation des shaders
\item Portabilité Windows/Linux
\end{itemize}
\item Difficulté algorithmique %Bon terme ?
\begin{itemize}[label=$\circ$]
\item Sélection et adaptation d'algorithmes
\end{itemize}
\end{itemize}
\begin{tabular}{l|r}
\includegraphics[scale=0.315]{rendu_sans_shader.png}
&
\includegraphics[scale=0.20]{singe_shaders.png}
\end{tabular}

\end{frame}
%------------------------------------------------

%\begin{frame}
%\frametitle{Gestion de projet}
%\begin{itemize}[label=$\bullet$]
%\item Cahier des charges
%	\begin{itemize}[label=$\circ$]
%	\item Recherche de l'existant
%	\item Définition des besoins fonctionnels et non fonctionnels
%	\item Prototypes de tests
%	\end{itemize}
%\end{itemize}
%\end{frame}

%------------------------------------------------
\section{Gestion de projet}

\begin{frame}
\frametitle{Déroulement du projet}
\begin{itemize}[label=$\bullet$]
\item Réalisation du cahier des charges
	\begin{itemize}[label=$\circ$]
	\item Rigueur de la rédaction
	\end{itemize}
\item Implémentation du logiciel
	\begin{itemize}[label=$\circ$]
	\item Répartition de l'équipe
	\item Utilisation des créneaux pour les réunions
	\end{itemize}
\end{itemize}

\end{frame}

%------------------------------------------------

\begin{frame}
\frametitle{Gestion de projet}
\begin{itemize}[label=$\bullet$]
\item Réalisation du diagramme de Gantt
\end{itemize}
{\fontsize{7}{8}\selectfont
\arrayrulecolor{black}
\begin{tabular}{lcr}

\textbf{Tâche} & \textbf{Durée Prévisionnelle} & \textbf{Durée réelle} \\
\\
\hline
\\
Recherche et implémentation \\des algorithmes &
4 semaines & 13 semaines \\
\hline
\\
Insertion de plusieurs objets \\ dans une scène &
3 semaines en fin de projet & 
1 semaine et demie \\ inclus dans 
une autre tâche \\ en début de projet \\
\hline
\\
Génération des flipbooks & 2semaines & 2 semaines \\
\hline
\\
Sauvegarde et chargement & 2 semaines puis 3 semaines &
7 semaines dont 4 effectives \\
\hline
\\
Prise de décision pour les shaders & X & 3 semaines\\

\end{tabular}
}

\end{frame}

%------------------------------------------------

\begin{frame}
\frametitle{Gestion de projet}
\begin{itemize}[label=$\bullet$]
\item Intérêts de ce projet
	\begin{itemize}[label=$\circ$]
	\item Projet de taille importante et équipe nombreuse
	\item Suivi régulier du client
	\end{itemize}
\item Quelques difficultés
	\begin{itemize}[label=$\circ$]
	\item Appréciation de la durée des tâches
	\item Gestion d'équipe
	\end{itemize}
\end{itemize}

\end{frame}

%%------------------------------------------------
%
% faire une transition pour la bliblioraphie
\frame[allowframebreaks]{\frametitle{Bibliographie}
	\bibliographystyle{unsrtnat}
	\bibliography{bibli}
		
	}
	
%------------------------------------------------
	% juste dire à l'oral merci pour votre attention
	% TODO : annuler le style de cette slide et laisse l'image en grand 
\begin{frame} 
\begin{figure}
\hspace*{-1cm}
\centering
\includegraphics[scale=0.6]{happy_dubois.png}
\end{figure}

\end{frame}

%----------------------------------------------------------------------------------------

\end{document} 
